\documentclass{article}
\usepackage[utf8]{inputenc}

\title{Positivity-preserving adaptive Runge-Kutta methods: project context and outline}
\author{}

\begin{document}

\maketitle


\section{Main idea}

The goal of this project is to develop a class of Runge--Kutta methods that enforce positivity or similar bound constraints (such as a maximum principle) on the numerical solution by adaptively choosing the method coefficients.

If one considers the stages and their coefficients to be fixed,
then choosing the weights in a way that guarantees a positive
solution is a linear programming problem.

\subsection{Additional potential directions}
\begin{enumerate}
    \item Adapt subquadrature weights ($A$) as well
    \item Adapt time step size as well
    \item Linear multistep methods
\end{enumerate}

\section{Challenges and potential solutions}
\begin{enumerate}
    \item Solving a linear programming problem at each step may be expensive.  This can be alleviated by pruning the set of positivity constraints.  Or, by focusing on implicit methods,
    this cost is likely to be less significant in relative terms.
    \item It is difficult to ensure stability of the resulting method,
    since stability properties depend in a highly nonlinear way
    on the weights.  One way to potentially avoid instabilities is
    to ensure that the weights are not too far from those of a known
    good method.
    \item There may be no solution to the linear programming problem.
    In this case we need a fallback.  This could involve resorting
    to a lower-order embedded method (and possibly modifying its
    weights as well).
\end{enumerate}

\section{Existing related approaches and their limitations}
\subsection{Strong stability preserving (SSP) methods}
Guarantee positivity under a step size restriction proportional
to the explicit Euler positive step size.  If that step size
is very small, SSP methods are not so useful.  Explicit
one-step SSP methods can only be fourth-order accurate.  Also,
implicit SSP methods do not guarantee positivity under
large step sizes.  See e.g. \cite{gottlieb2009high,gottlieb2011strong}.

\subsection{Modified Patankar methods}
Guarantee positivity for certain types of ODEs
(conservative production-destruction systems).
Not applicable outside that problem class.  These methods
are implicit, though only in a linear way.
It's difficult to achieve higher than 3rd-order
accuracy with them.  See e.g. \cite{ortleb2017patankar,kopecz2018unconditionally}.


\subsection{Spatial adaptation of the time discretization}
Yet another approach is to use different time integration methods
in a localized spatial region where positivity is an issue.
For instance, one can use a convex combination of high-
and low-order schemes, with the combination chosen locally
to ensure positivity \cite{duraisamy2003concepts}.
See also \cite{ketcheson2013spatially,duraisamy2007implicit}.

\subsection{Local adaptation of the spatial discretization}
There are many works that take this approach, but we won't
review them here as they are quite different in principle
from the present focus.

\section{Applications and test problems}

\subsection{Explicit methods}

\begin{enumerate}
    \item Scalar conservation laws (advection, Burgers): solution bounded by extrema of initial data
    \item Shallow water equations: enforce positive depth
    \item Euler equations: enforce positive density, pressure
    \item Reactive Euler equations or multiphase equations: box bounds
\end{enumerate}

\subsection{Implicit methods}

\begin{enumerate}
    \item Heat equation with non-smooth initial data: enforce positivity and/or maximum principle
    \item Robertson problem
    \item HV atmospheric pollution problem
\end{enumerate}

\section{Methods to work with}
\subsection{Explicit methods}

\begin{enumerate}
    \item  SSP(10,4): lots of degrees of freedom in choosing weights
    \item Other high-order explicit methods with extra stages
    \item  Explicit extrapolation or deferred correction methods: have a built-in hierarchy of methods of increasing order, to use as fallbacks.  In particular, extrapolation methods based on explicit Euler can be guaranteed to have positive intermediate stages.
\end{enumerate}

\subsection{Implicit methods}
\begin{enumerate}
    \item  Implicit extrapolation or deferred correction methods: have a built-in hierarchy of methods of increasing order, to use as fallbacks.  In particular, extrapolation methods based on implicit Euler can be guaranteed to have positive intermediate stages.
\end{enumerate}
\end{document}


\section{Research tasks}
The research for this project mainly consists of implementing
different approaches and testing the implementation on
appropriate problems.  At each step, depending on the test
results, various additional steps may be necessary.

\begin{itemize}
    \item Develop basic implementation of the idea and test it on simple problems.  Identify any problems that arise. (done)
    \item Test fallback to lower-order embedded methods.
    \item Implement implicit Euler extrapolation methods.
\end{itemize}
