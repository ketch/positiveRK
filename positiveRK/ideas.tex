\documentclass{article}

\usepackage{amsmath}

\title{Ideas for using Runge--Kutta methods to preserve positivity and similar properties}


\begin{document}
\maketitle

\section{Basic relaxation approach}

\section{Linear programming approach}
Given a Runge--Kutta method with coefficients $(A,b)$, after we have computed
the stages
$$
    y_j = u^n + h \sum_i a_{ij} f(y_i),
$$
the idea is to adaptively choose the weight $b_j$ in order to enforce positivity.
This leads to a linear program with the linear inequality constraints
$$
u^{n+1}_i = u^n_i + h \sum_j b_j f(y_j) \ge 0
$$
and the linear equality constraints that come from the order conditions.  The idea is
to enforce as many of these as possible.  For first order:
\begin{align}
\sum_j b_j = 1.
\end{align}
For second order:
\begin{align}
\sum_j c_j b_j = 1/2.
\end{align}
For third order:
\begin{align}
\sum_j c_j^2 b_j & = 1/3 \\
\sum_j \tau_{2j} b_j & = 0.
\end{align}
For fourth order:
\begin{align}
\sum_j c_j^3 b_j & = 1/4 \\
\sum_j \tau_{2j} c_j b_j & = 0 \\
\tau_{2}^T A^T b & = 0 \\
\tau_{3}^T b & = 0.
\end{align}
For fifth order:
\begin{align}
\tau_2^T A^T C b & = 0 \\
\tau_2^T C A^T b & = 0 \\
\tau_2^T (A^T)^2 b & = 0 \\
\tau_2^T C^2 b & = 0 \\
\tau_2^2 b & = 0 \\
\tau_3^T A^T b & = 0 \\
\tau_3^T C b & = 0 \\
(\tau_2^T)^2 b & = 0 \\
\tau_4^T b & = 0.
\end{align}
Here vector powers mean componentwise multiplication while
matrix powers mean matrix multiplication.  The matrix $C$ is the
diagonal matrix of the abscissas $c_i$ and
$\tau_k$ are the stage order residual vectors:
$$
    \tau_k - \frac{1}{k!} c^k - \frac{1}{(k-1)!}Ac^{k-1}.
$$
Since they only appear in equations where the right-hand-side is zero,
it's okay to multiply by $k!$ and use instead the vectors
$$
    \hat{\tau}_k - c^k - kAc^{k-1}.
$$




\end{document}
